\documentclass[letter,openright,12pt,spanish]{report}
%Gummi|065|=)
\title{\textbf{Descripci\'on de la conveci\'on de Denavit-Hartenberg}}
\author{Cinematica de Robots\\
		Nadia Sarahi Murgu\'ia Ch\'avez\\
		Ing. Mecatr\'onica 7A}
\date{}
\usepackage{amsmath}
\usepackage{graphicx}
\begin{document}

\maketitle

\section{Convenci\'on de Denavit-Hartenberg}

En el estudio de la rob\'otica, exite un algoritmo, llamado algoritmo de Denavit-Hartenberg, que nos ayuda a establecer los sitemas de referencia para cada uno de los eslabones con los que cuenta el robot.\\
Se trata de un procedimiento sistem\'atico para describir la estructura cinem\'atica de una cadena articulada constituida por articulaciones con un solo grado de libertad.\\
Para ello, a cada articulaci\'on se le asigna un \textbf{Sistema de Referencia Local} con origen en un punto \textbf{Q} y ejes ortonormales \textbf{[X, Y, Z]}, comenzando con un primer SR (Sistema de Referencia) fijo e inmovil dado por los ejes:

\begin{displaymath}
[\textbf{X}_0, \textbf{Y}_0, \textbf{Z}_0]
\end{displaymath} 
Anclado a un punto fijo \textbf{Q} de la \textbf{Base} sobre la que est\'a montada toda la estructura de la cadena.\\
Este sistema de refrencia no tiene por que ser el UNiversal con origen en \textbf{(0, 0, 0)} y la base can\'onica.

\subsection{Asignaci\'on de Sistemas de Referencia}

Las articulaciones se numeran desde \textbf{1} hasta \textbf{n}. A la articulacion \textbf{i}-\'esima se le asocia su propuo eje de rotaci\'on con eje:

\begin{displaymath}
\textbf{Z}_i-1
\end{displaymath}

De forma que el eje de giro de la 1ra articulaci\'on es:

\begin{displaymath}
\textbf{Z}_0
\end{displaymath}

y en la la \textbf{n}- \'esima articulaci\'on:

\begin{displaymath}
\textbf{Z}_n-1
\end{displaymath}
En la figura 1 se muestra la estructura del robot PUMA junto con sus articulaciones y ejes de rotaci\'on.
Para la articulaci\'on \textbf{i}-\'esima (que es la que gira alrededor de \textbf{Z}), la elecci\'on del origen de coordenadas \textbf{Q} y del eje \textbf{X} sigue reglas muy precisas en funci\'on de la geometr\'ia de los brazos articulados. El eje \textbf{Y} por su parte, se escoge para que el sistema{\textbf{X, Y, Z}} se dextr\'ogiro\footnote{}. La especificaci\'on de cada eje \textbf{X} depende de la relaci\'on espacial entre \textbf{Z} y \ textbf{Z}, dsitingui\'endose 2 cosaso:

\begin{figure}[htp]
\centering
\includegraphics[scale=1.00]{/home/sarha13/Escritorio/robot.png}
\caption{Robot PUMA}
\label{Figura 1.}
\end{figure}

\subsubsection{Z y Z no son paralelos}

Entonces existe una \'unica recta perpendicular a ambos, cuya intersecci\'on con los ejes proporciona su m\'inima distancia (que puede ser 0). Esta distancia, \textbf{a}, medida desde el eje \textbf{Z} hacia el eje \textbf{Z} (con su signo), es uno de los par\'ametros asociados a la articulaci\'on \textbf{i}-\'esima. 
La distancia \textbf{d} desde \textbf{Q} a la intersecci\'on de la perpendicular com\'un entre \textbf{Q} a la intersecci\'on de la perpedicular com\'un entre \textbf{Z} y \textbf{Z} con \textbf{Z} es el 2do de los par\'ametros.
En la figura 2, podemos denotar que el eje \textbf{X} es una recta, siendo el sentido positivo el que va desde el eje \textbf{Z} al \textbf{Z} si \textbf{a>0}.
El origen de coordenadas \textbf{Q} es la intesecci\'on de dicha recta con ele \textbf{Z}.

\begin{figure}[htp]
\centering
\includegraphics[scale=1.00]{/home/sarha13/Escritorio/plano.png}
\caption{Planos}
\label{Figura 2.}
\end{figure}

\subsubsection{Z y z son paralelos}

En esta situaci\'on el eje \textbf{X} se toma en el plano conteniendo a \textbf{Z} y \textbf{Z} y perpendicular a ambos. 
El origen \textbf{Q} es cualquiera punto conveniente del eje \textbf{Z}. 
El par\'ametro \textbf{a} es, como antes, la distancia perpendicular entre los ejes \textbf{Z} y \textbf{Z} y \textbf{d} es la distancia desde \textbf{Q}.
Una vez determinado el eje \textbf{X}, a la articulaci\'on \textbf{i}-\'esima se le asocia un 3er par\'ametro fijo \textbf{a} que es el \'angulo que forman los ejes \textbf{Z} y \textbf{Z} en relaci\'onal eje \textbf{X}.\\
N\'otese en la figura 3, que cuando el brazo \textbf{i}-\'esimo (que une rigidamente las articulaciones \textbf{i} e \textbf{i+1}) gira en torno al eje \textbf{Z} (que es el de rotaci\'on de la articulaci\'on \textbf{i}), los par\'ametros \textbf{a}, \textbf{d} y \textbf{a} permamnecen cosntantes, pues depende exclusivamente de las posiciones/orientaciones relativas entre los ejes \textbf{Z} y \textbf{Z}, que son invariables. Por tanto, \textbf{a}, \textbf{d} y \textbf{a} pueden calcularse a partir de cualquier configuraci\'on de la esturctura articulada, en particular a partir de una configuraci\'on inicial est\'andar. Precisamente el \'angulo $\theta$, de giro que forman los ejes \textbf{X} y \textbf{X} con respecto al eje \textbf{Z} es el 4to par\'ametro asociado a la articulaci\'on \textbf{i} y el unico de ellos que varia cuando el brazo \textbf{i} gira.
Es importente observar que le conjunto de los 4 par\'ametros \textbf{a}, \textbf{d}, \textbf{a} y \textbf{$\theta$} determina totalmente el Sistema de Referencia de la articulaci\'on \textbf{i+1} en funci \'on del S.R de la articulaci\'on \textbf{i}.

\begin{figure}[htp]
\centering
\includegraphics[scale=1.00]{/home/sarha13/Escritorio/coordenadas.png}
\caption{Coordenadas}
\label{Figura 3.}
\end{figure}

\section{Transformaci\'on de coordenadas}

De los 4 par\'ametros aosciados a una articulaci\'on, los 3 primeros son constantes y depende exclusivamente de la relaci\'on geometr\\iaca entre las articulaciones \textbf{i} e \textbf{i+1}, mientras que el 4to par\'ametro $\theta$ es la \'unica variable de la articulaci\'on, siendo el \'angulo de giro del eje \textbf{X} alrededor del eje \textbf{Z} para llevarlo hasta \textbf{X}. 
Sabemos que dados 2 Sistemas de Referencia 

\begin{displaymath}
\textbf{R}_1\textbf{= Q}_1 \textbf{,[u}_1\textbf{,u}_2\textbf{,u}_3\textbf{]}\\
\end{displaymath}

\begin{displaymath}
\textbf{R}_2\textbf{=Q}_2\textbf{.[v}_1\textbf{,v}_2\textbf{,v}_3\textbf{]}\\
\end{displaymath}
con bases ortonormales asociadas, el cambio de coordenadas del segundo S.R al primero viene dado por:

\begin{figure}[htp]
\centering
\includegraphics[scale=1.50]{/home/sarha13/Escritorio/ecuacion1.png}
\caption{matriz}
\label{Figura 4.}
\end{figure}
donde:

\begin{displaymath}
\beta_1, \beta_2, \beta_3
\end{displaymath}
son las coordenadas de un punto en el S.R \textbf{\begin{displaymath}R_2, R\end{displaymath}}
es la matriz del Cmabio de Base tal que:

\begin{displaymath}
\textbf{[v}_1|\textbf{v}_2|\textbf{v}_3\textbf{]} = \textbf{[u}_1|\textbf{u}_2|\textbf{u}_3\textbf{]}\cdot\textbf{R}\\
\end{displaymath}

\begin{displaymath} 
\textbf{$\lambda$}_1,\textbf{$\lambda$}_2, \textbf{$\lambda$}_3
\end{displaymath}
son las coordenadas del origen del segundo S.R., \textbf{Q} respecto al primero. La expresi\'on permite entonces obtener las coordenadas:

\begin{displaymath}
\textbf{a}_1, \textbf{a}_2, \textbf{a}_3
\end{displaymath}

del punto en cuesti\'on con respecto al primero de los S.R.
En nuestro caso, para pasar de la \textbf{(i+1)}-\'esima articulaci\'on a la \textbf{i}-\'esima, los Sistemas de Referencia son:

\begin{displaymath}
\textbf{R}_1={\textbf{Q}_i-1,\textbf{[X}_i-1, \textbf{Y}_i-1, \textbf{Z}_i-1 \textbf{]}}
\end{displaymath}
\begin{displaymath}
\textbf{R}_2={\textbf{Q}_i,\textbf{[X}_i,\textbf{Y}_i.\textbf{Z}_1\textbf{]}}
\end{displaymath}
Estudiaremos por separado la matriz del Cambio de Base y la expresi\'on de \textbf{Q} en el pierm S.R.

\subsection{Matriz del Cambio de Base}

Haciendo asignado los ejes a cada articulaci\'on mediante la representaci\'on Denavit-Hartenberg, tenemos que:

1- El eje \textbf{X} se obtiene rotando eleje \textbf{X} alrededor del eje \textbf{Z} un \'angulo $\theta$.

2-El eje \textbf{Z} se obtiene rotando el eje \textbf{Z} añrededor del eje \textbf{X} un \'angulo \textbf{$\alpha$}.

Por su parte, el eje \textbf{Y} viene ya determinado por \textbf{X} y \textbf{Z}.

La primera transformaci\'on es una rotaci\'on alredeor del 3er vector de la 1ra Base, cuyas ecuaciones gen\'ericas son:

\begin{displaymath}
[u_1^{(1)}|u_2^{(1)}|u_3^{(1)}=[u_1|u_2|u_3]R_3(\theta_1)
\end{displaymath}

La segunda transofmraci\'on es una rotaci\'on alrededor del 1er vector de la base ya transformada y tiene por expresi\'on:

\begin{displaymath}
[u_1^{(2)}|u_2^{(2)}|u_3^{(2)}]=[u_1^{(1)}|u_2^{(1)}|u_3^{(1)}]R_1(\alpha_i)
\end{displaymath}
Por tanto, concaten\'andolas:

\begin{displaymath}
[u_1^{(2)}|u_2^{(2)}|u_3^{(2)}]=[u_1|u_2|u_3]R_3(\theta_i)R_1(\alpha_i)
\end{displaymath}
Finalmente, cambiamos la notaci\'on para tener:

\begin{displaymath}
[X_i|Y_i|Z_i]=[X_{i-1}|Y_{i-1}|Z_{i-1}]R_3(\theta_i)R_1(\alpha_i)
\end{displaymath}
Con lo cual, la matriz del cambio de base es:

\begin{figure}[htp]
\centering
\includegraphics[scale=1.00]{/home/sarha13/Escritorio/ecuaicon6.png}
\caption{Matriz 2}
\label{}
\end{figure}

\begin{figure}[htp]
\centering
\includegraphics[scale=1.00]{/home/sarha13/Escritorio/matriz.png}
\caption{Matriz 2.1}
\label{}
\end{figure}


\subsubsection{Coordenadas de Q en el primer S.R}

Seg\'un la representaci\'on de Denavit-Hartenberg el origen del 2do Sistema de Referencia se obtiene mediante:

1- Traslaci\'on de \textbf{Q} a lo largo del eje \textbf{Z} por la magnitud \textbf{d}.

2- Traslaci\'on a lo largo del eje \textbf{X} por la magnitud \textbf{$\alpha$}

La pirmera transformaci\'on es:

\begin{displaymath}
Q_{i-1}^{(1)}=Q_{i-1}+d_iZ_{i-1}
\end{displaymath}

La segunda transformacii\'on es:\\

\begin{displaymath}
Q_i=Q_{i-1}^{(1)}+\alpha_iX_i
\end{displaymath}


Teniendo ahora en cuenta que:\\

\begin{figure}[htp]
\centering
\includegraphics[scale=1.00]{/home/sarha13/Escritorio/ecuacion2.png}
\caption{}
\label{}
\end{figure}

Se tiene el 1er vector como:\\

\begin{figure}[htp]
\centering
\includegraphics[scale=1.00]{/home/sarha13/Escritorio/ecuacion3.png}
\caption{}
\label{}
\end{figure}

de donde:

\begin{figure}[htp]
\centering
\includegraphics[scale=1.00]{/home/sarha13/Escritorio/ecuacion4.png}
\caption{}
\label{}
\end{figure}

\begin{displaymath}
Q_i=Q_{i-1}^{(1)}+\alpha_iX_i=Q_{i-1}+d_iZ{i-1}+\alpha_i(cos\theta_i\cdot X_{i-1}+sen\theta_1\cdot Y_{i-1})
\end{displaymath}

\begin{displaymath}
Q_i=Q_{i-1}+(\alpha_icos\theta_i)X_{i-1}+(\alpha_isen\theta_i)Y_{i-1}+d_iZ_{i-1}
\end{displaymath}

y por tanto, las coordenadas de ,\textbf{Q} es el 1er Sistema de Referencia son:\\\\\\\\\\\\

\begin{figure}[htp]
\centering
\includegraphics[scale=1.00]{/home/sarha13/Escritorio/matriz3.png}
\caption{}
\label{}
\end{figure}


Finalmente, la transformaci\'on de coordenadas del S.R. \textbf{Q, [X, Y, Z]} al S.R. \textbf{Q, [X, Y, Z]} es:\\

\begin{figure}[htp]
\centering
\includegraphics[scale=1.00]{/home/sarha13/Escritorio/ecuacion4.png}
\caption{}
\label{}
\end{figure}

Cambiando la notaci\'on para las coordenadas:\\

\begin{figure}[htp]
\centering
\includegraphics[scale=1.00]{/home/sarha13/Escritorio/ecuacion5.png}
\caption{}
\label{}
\end{figure}

Donde el subindice denota el Sistema de Referencia respecto al cual est\'an expresadas las coordenadas.\\\\

\newpage

En coordenadas homog\'eneas:

\begin{figure}[htp]
\centering
\includegraphics[scale=1.00]{/home/sarha13/Escritorio/matriz2.png}
\caption{}
\label{}
\end{figure}
\newpage

\begin{thebibliography}{X}
\bibitem{Barrientos} \textbf{Barrientos A.;}  \textbf{Peñin L. F.;} \textbf{Balaguer C.$\&$ Aracil R.} \textsc{Fundamentos de Robotica} 2° Ed. McGraw-Hill, 2007

\bibitem{FU} \textbf{FU, K. S.; G\'onzalez, R. C. $\&$ Lee, C. S. G.} \textsc{Robotica: control, deteccion, vision e inteligencia} mcgraw-hill, 1998
\end{thebibliography}
\end{document}
